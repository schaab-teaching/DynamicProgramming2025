\documentclass[11pt, aspectratio=169]{beamer}

\usepackage{amsmath, amsfonts, microtype, nicefrac, amssymb, amsthm, centernot}

\usepackage{pgfpages}

\usepackage{helvet}
\usepackage[default]{lato}
\usepackage{array}

\usefonttheme[onlymath]{serif}

\usepackage[utf8]{inputenc}
\usepackage[T1]{fontenc}
\usepackage{textcomp}
\usepackage{bm}

\usepackage{mathpazo}
\usepackage{hyperref}
\usepackage{multimedia}
\usepackage{graphicx}
\usepackage{multirow}
\usepackage{graphicx}
\usepackage{dcolumn}
\usepackage{bbm}
\newcolumntype{d}[0]{D{.}{.}{5}}

\usepackage{graphicx}
\usepackage[space]{grffile}
\usepackage{booktabs}

\usepackage{setspace}

\usepackage{transparent}


%%% FIGURES %%%
\usepackage{caption, subcaption}
\usepackage{booktabs, siunitx}
\usepackage{pgfplots} 
%\usepackage[outdir=./figures]{epstopdf}
\usepackage{float}
\usepackage{graphicx}
\usepackage[absolute, overlay]{textpos}
\usepackage{epstopdf}


%%% TIKZ %%%
\usepackage{tikz}
\usepackage{verbatim}
\usetikzlibrary{arrows.meta}
\usetikzlibrary{positioning}
\usetikzlibrary{bending}
\usetikzlibrary{snakes}
\usetikzlibrary{calc}
\usetikzlibrary{arrows}
\usetikzlibrary{decorations.markings}
\usetikzlibrary{shapes.misc}
\usetikzlibrary{matrix, shapes, arrows, fit, tikzmark}


%%% ALGORITHM %%%
\usepackage{algorithm}
\usepackage[noend]{algpseudocode}
\usepackage{multimedia}


%%% APPENDIX SLIDE NUMBERING %%%
\usepackage{appendixnumberbeamer}


%%% BEAMER BUTTON %%%
%\setbeamertemplate{button}{\tikz
	%\node[
	%	inner xsep = 2pt, 
	%	draw = structure!0, 
	%	fill = myblue, 
	%	rounded corners = 4pt]{\color{white} \tiny\insertbuttontext};
	%}


%%% COLORS %%%
\definecolor{blue}{RGB}{0,38,118}
\definecolor{red}{RGB}{213,94,0}
\definecolor{yellow}{RGB}{240,228,66}
\definecolor{green}{RGB}{0,158,115}

\definecolor{myred}{RGB}{163,32,45}
\definecolor{navyblue}{rgb}{0.05,0.2,0.70}
\definecolor{myblue}{RGB}{0,51,150}
\definecolor{myorange}{RGB}{255,140,0}
\definecolor{myref}{RGB}{160,160,160}
\definecolor{shock}{RGB}{0, 125, 34}%{50, 168, 82}

\definecolor{background}{RGB}{255,253,218}

% Define a new transparent color
\definecolor{trans}{rgb}{1,1,1}
\colorlet{trans}{black!20} % 0 percent opacity

\hypersetup{
  colorlinks=false,
  linkbordercolor = {white},
  linkcolor = {blue}
}

\setbeamercolor{frametitle}{fg=blue}
\setbeamercolor{title}{fg=black}
\setbeamertemplate{footline}[frame number]
\setbeamertemplate{navigation symbols}{} 
\setbeamertemplate{itemize items}{-}
\setbeamercolor{itemize item}{fg=blue}
\setbeamercolor{itemize subitem}{fg=blue}
\setbeamercolor{enumerate item}{fg=blue}
\setbeamercolor{enumerate subitem}{fg=blue}
\setbeamercolor{button}{bg=background, fg=blue,}

%\setbeamercolor{background canvas}{bg=background}


%%% FRAME TITLE %%%
\setbeamerfont{title}{series=\bfseries, parent=structure}
\setbeamerfont{frametitle}{series=\bfseries, parent=structure}


%%% TRANSITION FRAME %%%
\newenvironment{transitionframe}{
	\setbeamercolor{background canvas}{bg=blue}
	\begin{frame}
		\thispagestyle{empty}
		\addtocounter{framenumber}{-1}
		\vspace{42mm}
		\hspace{4mm} }{
		\begin{tikzpicture}
			\tikz \fill [white] (1,6) rectangle (20,10);
		\end{tikzpicture}
	\end{frame}
}


%%% OUTLINE %%%
\AtBeginSection[]
{
	\begin{frame}
       \frametitle{Roadmap of Talk}
       \tableofcontents[currentsection]
   \end{frame}
}
\setbeamercolor{section in toc}{fg=blue}
\setbeamercolor{subsection in toc}{fg=red}
\setbeamersize{text margin left=1em,text margin right=1em} 


%%% ENVIRONMENTS
\newenvironment{witemize}{\itemize\addtolength{\itemsep}{10pt}}{\enditemize}

\makeatother
\setbeamertemplate{itemize items}{\large\raisebox{0mm}{\textbullet}}
\setbeamertemplate{itemize subitem}{\footnotesize\raisebox{0.15ex}{--}}
\setbeamertemplate{itemize subsubitem}{\Tiny\raisebox{0.7ex}{$\blacktriangleright$}}

\setbeamertemplate{enumerate item}[default]
\setbeamertemplate{enumerate subitem}{\textbullet}
\makeatletter

% ITEMIZE SPACING:
% \usepackage{xpatch}
% \xpatchcmd{\itemize}
% {\def\makelabel}
% {\setlength{\itemsep}{0mm}\def\makelabel}
% {}
% {}


%%% PRETTY ENUMERATE %%%
% \usepackage{stackengine,xcolor}
% \newcommand\circnum[2]{\stackinset{c}{}{c}{.1ex}{\small\textcolor{white}{#2}}%
	% 	{\abovebaseline[-.7ex]{\Huge\textcolor{#1}{$\bullet$}}}}
% \newenvironment{myenum}
% {\let\svitem\item
	% 	\renewcommand\item[1][black]{%
		% 		\refstepcounter{enumi}\svitem[\circnum{##1}{\theenumi}]}%
	% 	\begin{enumerate}}{\end{enumerate}}
\usepackage{stackengine,xcolor,graphicx}
\newcommand\circnum[2]{\smash{\stackinset{c}{}{c}{.2ex}{\small\textcolor{white}{#2}}%
		{\abovebaseline[-1.1ex]{\Huge\textcolor{#1}{\scalebox{1.5}{$\bullet$}}}}}}
\newenvironment{myenum}
{\let\svitem\item
	\renewcommand\item[1][black]{%
		\refstepcounter{enumi}\svitem[\circnum{##1}{\theenumi}]}%
	\begin{enumerate}}{\end{enumerate}}

\newcommand{\notimplies}{\;\not\!\!\!\implies}



%%%%%%%%%%%%%%%%%%%%%%%%%%  TITLE   %%%%%%%%%%%%%%%%%%%%%%%%%%%%%%%%
\title[]{\\[8pt]
	{\large \color{blue} Dynamic Programming and Applications \\[5pt] \normalfont{Discrete Time Dynamics and Optimization} \\[10pt] \normalfont{Lecture 1}}}
\author[Schaab]{Andreas Schaab}
\institute{}
\subject{}
\date{}



%%%%%%%%%%%%%%%%%%%%%%%%  BEGIN DOC   %%%%%%%%%%%%%%%%%%%%%%%%%%%%%%%
\begin{document}

%%% TIKZ %%% 
\tikzstyle{every picture}+=[remember picture]
%\everymath{\displaystyle}

\tikzset{   
	every picture/.style={remember picture,baseline},
	every node/.style={anchor=base,align=center,outer sep=1.5pt},
	every path/.style={thick},
}
\newcommand\marktopleft[1]{%
	\tikz[overlay,remember picture] 
	\node (marker-#1-a) at (-.3em,.3em) {};%
}
\newcommand\markbottomright[2]{%
	\tikz[overlay,remember picture] 
	\node (marker-#1-b) at (0em,0em) {};%
}
\tikzstyle{every picture}+=[remember picture] 
\tikzstyle{mybox} =[draw=black, very thick, rectangle, inner sep=10pt, inner ysep=20pt]
\tikzstyle{fancytitle} =[draw=black,fill=red, text=white]


\addtocounter{framenumber}{-1}
\thispagestyle{empty}
\maketitle 
\newpage


%%%%%%%%%%%%%%%%%%%%%%%%%%  SLIDE   %%%%%%%%%%%%%%%%%%%%%%%%%%%%%%%%
\begin{frame}{Introduction}
\begin{witemize}
\item Many (most) economic phenomena are about dynamics

\item Dynamics $\sim$ changes in the behavior of economic agents over time

\item This is a course about \textbf{dynamic optimization}: main approach to study dynamic models in economics 

\item Goal of this course: hard skills + empower you to do research

$\implies$ \textit{focus on tools, but with tons of applications }

\item The more you can ``tool up'' during your first two years, the better

\item Specifically: \textbf{dynamic programming} one of most powerful tools in economic analysis

\end{witemize}
\end{frame}


%%%%%%%%%%%%%%%%%%%%%%%%%%  SLIDE   %%%%%%%%%%%%%%%%%%%%%%%%%%%%%%%%
\begin{frame}{Dynamic programming will empower you}
\begin{witemize}
\item Macro: consumption, investment, portfolio allocation, ... 

\vspace{-2mm}
\item Labor: search, wage bargaining, ... 

\vspace{-2mm}
\item IO: competition and games, pricing, ...

\vspace{-2mm}
\item Finance: asset pricing, dynamic capital structure, portfolio choice, ... 

\vspace{-2mm}
\item Growth: technology adoption, poverty traps, firm innovation, ... 

\vspace{-2mm}
\item Trade / Urban: migration in spatial models, ... 

\vspace{-2mm}
\item Inequality: evolution of income-wealth distribution, ... 

\vspace{-2mm}
\item Game theory: dynamic games, ... 

\vspace{-2mm}
\item ... much, much more 

\end{witemize}
\end{frame}



%%%%%%%%%%%%%%%%%%%%%%%%%%  SLIDE   %%%%%%%%%%%%%%%%%%%%%%%%%%%%%%%%
\begin{frame}{Acknowledgements}

\begin{witemize}
\item Course builds on excellent teaching material developed by others

\item Huge thanks to David Laibson and Benjamin Moll for making available their incredible teaching material

\url{https://projects.iq.harvard.edu/econ2010c/lecturesDLaibson}

\url{https://benjaminmoll.com/}

\item Grateful to many others whose material I build on: Pablo Kurlat, Gabriel Chodorow-Reich, Gianluca Violante, ...  

\item Only required textbook: LeVeque on finite difference methods (check syllabus for many other recommendations)
\end{witemize}
\end{frame}


%%%%%%%%%%%%%%%%%%%%%%%%%%  SLIDE   %%%%%%%%%%%%%%%%%%%%%%%%%%%%%%%%
\begin{frame}{GitHub}
\begin{witemize}
\item One super useful \textit{hard skill} for research / code development: version control

\item You should start using git (via GitHub) in your own work from the start

\item Course material will be available via a GitHub organization at:
\url{https://github.com/schaab-teaching/DynamicProgramming2024}

\item You should create a GitHub account if you don't already have one

\item If this is new for you, I also recommend GitKraken

\item Get Pro versions of GitHub and GitKraken for free (verify your academic affiliation)

\item Great resource: \url{https://git-scm.com/book/en/v2} \\
(Especially chapters 2, 3, 5, and 6)

\end{witemize}
\end{frame}


%%%%%%%%%%%%%%%%%%%%%%%%%%  SLIDE   %%%%%%%%%%%%%%%%%%%%%%%%%%%%%%%%
\begin{frame}{Admin}
\begin{witemize}
\item Course co-taught by Kiyea and myself

\item Lectures will focus on theory and applications; sections will teach you coding

\item Problem sets every week. Two parts in two separate (!) docs: analytical and numerical. They are optional. Final counts for 100\% of your grade. 

{\footnotesize \textit{You are responsible from now on}}

\item {\color{blue} \textbf{However}}: completing the homework earns you extra credit, up to 60\%

{\footnotesize To illustrate, suppose you score 75/100 on exam. Without homework, 75\%. If you complete half of the problems each week: 30 + 0.7*75 = 82.5\%. If you complete all problems: 60 + 0.4*75 = 90\%.} 

\item Final exam will be close to the homework: I want to maximally incentivize you to work hard on the homework

\end{witemize}
\end{frame}


%%%%%%%%%%%%%%%%%%%%%%%%%%  SLIDE   %%%%%%%%%%%%%%%%%%%%%%%%%%%%%%%%
\begin{frame}{Honor Code}

Rules of the game:

\vspace{5mm}
\begin{quote}
	By turning in homework to receive extra credit, I affirm that (i) I did not consult past solution keys and (ii) I intellectually contributed to each problem answer I am submitting.
\end{quote}

\end{frame}

%%%%%%%%%%%%%%%%%%%%%%%%%%  SLIDE   %%%%%%%%%%%%%%%%%%%%%%%%%%%%%%%%
\begin{frame}{Course Overview}

Part I: dynamic programming (with examples and applications)
\begin{witemize}
\vspace{1mm}
\item Lectures 1-2: Dynamic programming in discrete time
\vspace{-2mm}
\item Lectures 3-4: Deterministic dynamic programming in continuous time
\vspace{-2mm}
\item Lectures 5-6: Stochastic dynamic programming in continuous time
\end{witemize}

\vspace{5mm}
Part II: (even more) applications
\begin{witemize}
\vspace{1mm}
\item Lectures 7-8: Consumption
\vspace{-2mm}
\item Lectures 9: Investment
\vspace{-2mm}
\item Lectures 10-11: Inequality
\vspace{-2mm}
\item Lectures 12-13: Welfare 
\end{witemize}
\end{frame}



%%%%%%%%%%%%%%%%%%%%%%%%%%  SLIDE   %%%%%%%%%%%%%%%%%%%%%%%%%%%%%%%%
\begin{frame}{Outline}
\thispagestyle{empty}
\addtocounter{framenumber}{-1}

Part 1: Neoclassical growth model
\begin{enumerate}
	\item Environment
	\item Planning problem
\end{enumerate}

\vspace{5mm}
Part 2: dynamic programming 
\begin{enumerate}
	\item Bellman equation
	\item First-order and envelope conditions
	\item Solving the Bellman equation with guess-and-verify
	\item Bellman operator 
\end{enumerate}

\vspace{5mm}
Part 3: Competitive equilibrium

\end{frame}






%%%%%%%%%%%%%%%%%%%%%%%%%%  SLIDE   %%%%%%%%%%%%%%%%%%%%%%%%%%%%%%%%
\begin{transitionframe}
	{\color{white} \Huge \textbf{Part 1: Neoclassical Growth Model} \vspace{2mm}}
\end{transitionframe}


%%%%%%%%%%%%%%%%%%%%%%%%%%  SLIDE   %%%%%%%%%%%%%%%%%%%%%%%%%%%%%%%%
\begin{frame}{1. Environment}
\begin{witemize}
\item Time is discrete with $t = 0, 1, \ldots$ and there is no uncertainty

\item Consider a \textbf{representative household} 

\item \textbf{Preferences} over consumption $c_t$ given by
\begin{equation*}
	\max_{ \{ c_t \}_{t=0}^\infty } \sum_{t = 0}^\infty \beta^t u(c_t)
\end{equation*}

\item $u(\cdot)$ is \textbf{flow utility}: units of $c_t$ ``dollars'' or ``apples'', units of $u(c_t)$ ``utils''
\end{witemize}
\end{frame}




%%%%%%%%%%%%%%%%%%%%%%%%%%  SLIDE   %%%%%%%%%%%%%%%%%%%%%%%%%%%%%%%%
\begin{frame}{}
\begin{witemize}
\item \textbf{Technologies}: final good produced according to 
\begin{equation*}
	y_t = f(k_t)
\end{equation*}
and capital accumulation technology is 
\begin{equation*}
	k_{t+1} = i_t + (1 - \delta) k_t 
\end{equation*}
for given $k_0 = \bar k$ (\textbf{initial condition})

\item \textbf{Resource constraint:}
\begin{equation*}
	y_t = c_t + i_t
\end{equation*}
\end{witemize}

\vspace{4mm}
\textbf{Definition.} Given initial capital stock $k_0$, a \textbf{feasible allocation} consists of sequences $\{y_t, c_t, i_t, k_t\}_{t = 0}^\infty$ that satisfy the production and capital accumulation technologies, the resource constraint, as well as non-negativity constraints $y_t \geq 0$, $c_t \geq 0$, $i_t \geq 0$ and $k_{t+1} \geq 0$.

\end{frame}


%%%%%%%%%%%%%%%%%%%%%%%%%%  SLIDE   %%%%%%%%%%%%%%%%%%%%%%%%%%%%%%%%
\begin{frame}{2. Planning problem}

\vspace{4mm}
\textbf{Definition.} The planning problem is to choose a feasible allocation that maximizes the lifetime utility of the representative household.  

\vspace{5mm}
\begin{witemize}
\item Notice: we have not introduced prices, markets, budget constraints, etc.

\item Why is planning problem relevant? (i) important in its own right to understand efficiency, and (ii) turns out first welfare theorem applies here, so efficient allocation = competitive equilibrium allocation

\item Social welfare = lifetime utility of representative household (why?)

\item ``choose a feasible allocation'' = choose sequences $\{y_t, c_t, i_t, k_t\}_{t = 0}^\infty$ subject to technologies and resource constraints
\end{witemize}
\end{frame}



%%%%%%%%%%%%%%%%%%%%%%%%%%  SLIDE   %%%%%%%%%%%%%%%%%%%%%%%%%%%%%%%%
\begin{frame}{}
\begin{witemize}
\item Assume $f(k_t) = k_t^\alpha$ with $\alpha \in (0, 1)$ and $\delta = 1$

\item Can write the planning problem as 
\begin{equation*}
	V(k_0) = \max_{ \{ c_t \}_{t=0}^\infty } \sum_{t = 0}^\infty \beta^t u(c_t)
\end{equation*}
s.t. 
\begin{equation*}
	k_{t+1} = k_t^\alpha - c_t. 
\end{equation*}

\item This is the \textbf{sequence problem} (or problem in sequence form)

\item We can substitute in: 
\begin{equation*}
	V(k_0) = \max_{ \{ k_{t+1} \}_{t=0}^\infty } \sum_{t = 0}^\infty \beta^t u(k_t^\alpha  - k_{t+1})
\end{equation*}

\item Given initial condition $k_0$

\end{witemize}
\end{frame}



%%%%%%%%%%%%%%%%%%%%%%%%%%  SLIDE   %%%%%%%%%%%%%%%%%%%%%%%%%%%%%%%%
\begin{frame}{}
\begin{itemize}
\item We can tackle the sequence problem directly using tools from constrained dynamic optimization

\item Lagrangian after substituting with FOC for $k_{t+1}$:
\begin{equation*}
	L(k_0) =  \sum_{t = 0}^\infty \beta^t u(k_t^\alpha  - k_{t+1})
\end{equation*}
\begin{equation*}
	0 = - \beta^t u'(c_t) + \beta^{t+1} u'(c_{t+1}) \alpha k_{t+1}^{\alpha - 1}
\end{equation*}

\item Lagrangian with multiplier before substituting with FOCs for $c_t$ and $k_{t+1}$:
\begin{equation*}
	L(k_0) =  \sum_{t = 0}^\infty \beta^t \bigg[ u(c_t) + \lambda_t \bigg( k_t^\alpha - c_t - k_{t+1} \bigg) \bigg]
\end{equation*}
\begin{align*}
	0 &= \beta^t u'(c_t) - \beta^t \lambda_t \\
	0 &= -\beta^t \lambda_t + \alpha \beta^{t+1} \lambda_{t+1} k_{t+1}^{\alpha - 1}
\end{align*}

\end{itemize}
\end{frame}



%%%%%%%%%%%%%%%%%%%%%%%%%%  SLIDE   %%%%%%%%%%%%%%%%%%%%%%%%%%%%%%%%
\begin{frame}{}

\textbf{Proposition.} An allocation $\{y_t, c_t, i_t, k_t\}_{t = 0}^\infty$ is (Pareto) efficient if it is feasible and satisfies   
\begin{equation*}
	u'(c_t) = \beta f'(k_{t+1}) u'(c_{t+1}).
\end{equation*}

\vspace{6mm}
In other words, an efficient allocation solves the equations:
\begin{align*}
	u'(c_t) &= \beta f'(k_{t+1}) u'(c_{t+1}) \\
	y_t &= f(k_t) \\
	k_{t+1} &= i_t \\
	y_t &= c_t + i_t,
\end{align*}
given initial condition $k_0 = \bar k$.

Always count equations and unknowns!!

\end{frame}




%%%%%%%%%%%%%%%%%%%%%%%%%%  SLIDE   %%%%%%%%%%%%%%%%%%%%%%%%%%%%%%%%
\begin{transitionframe}
	{\color{white} \Huge \textbf{Part 2: Dynamic Programming} \vspace{2mm}}
\end{transitionframe}



%%%%%%%%%%%%%%%%%%%%%%%%%%  SLIDE   %%%%%%%%%%%%%%%%%%%%%%%%%%%%%%%%
\begin{frame}{1. Bellman equation}

\textbf{Definition.} The \textbf{Bellman equation} characterizes the value function as the sum of the \textbf{flow payoff} and the discounted \textbf{continuation value}
\begin{equation*}
	V(k) = \max_{k'} \Big\{ u(k^\alpha - k') + \beta V(k') \Big\} \quad \text{ for all } k
\end{equation*}

\begin{witemize}
\item We call $V(k)$ the value function and $u(k^\alpha - k')$ flow payoff or (instantaneous) utility flow

\item \textbf{Recursive representation} of the planning problem (not sequence form)

\item We say that $\mathcal X = [0, \bar k]$ is the \textbf{state space} of the neoclassical growth model. The Bellman equation must hold for all feasible levels of capital $k \in \mathcal X$.

\item If $V(k)$ solves the above equation, then it is a solution to the Bellman equation. Next: the unique value function that solves sequence problem also solves Bellman equation

\end{witemize}
\end{frame}


%%%%%%%%%%%%%%%%%%%%%%%%%%  SLIDE   %%%%%%%%%%%%%%%%%%%%%%%%%%%%%%%%
\begin{frame}{}
\begin{witemize}
\item Sequence problem $\longrightarrow$ recursive representation (Bellman equation)
\begin{align*}
	V(k_0) &= \max_{ \{ k_{t+1} \}_{t=0}^\infty } \, \bigg\{  \sum_{t = 0}^\infty \beta^t u(k_t^\alpha  - k_{t+1}) \bigg\} \\
	&= \max_{ \{ k_{t+1} \}_{t=0}^\infty } \, \bigg\{ u(k_0^\alpha  - k_1) + \sum_{t = 1}^\infty \beta^t u(k_t^\alpha  - k_{t+1}) \bigg\}  \\
	&= \max_{ \{ k_{t+1} \}_{t=0}^\infty } \, \bigg\{ u(k_0^\alpha  - k_1) + \beta \sum_{t = 1}^\infty \beta^{t-1} u(k_t^\alpha  - k_{t+1}) \bigg\} \\
	&= \max_{k_1} \, \bigg\{ u(k_0^\alpha  - k_1) + \beta \max_{ \{ k_{t+1} \}_{t=1}^\infty } \, \bigg\{  \sum_{t = 0}^\infty \beta^t u(k_{t+1}^\alpha  - k_{t+2}) \bigg\} \bigg\} \\
	&= \max_{ k_1} \, \bigg\{ u(k_0^\alpha  - k_1) + \beta V(k_1) \bigg\}
\end{align*}

\item Recursively, with $k$ capital ``today'' and $k'$ capital ``tomorrow''
\begin{align*}
	V(k) &= \max_{k'} \, \bigg\{ u(k^\alpha  - k') + \beta V(k') \bigg\}
\end{align*}

\end{witemize}
\end{frame}


%%%%%%%%%%%%%%%%%%%%%%%%%%  SLIDE   %%%%%%%%%%%%%%%%%%%%%%%%%%%%%%%%
\begin{frame}{}
\begin{witemize}
\item A solution to the Bellman equation also solves Sequence Problem
\begin{align*}
	V(k_0) &= \max_{ k_1} \, \bigg\{ u(k_0^\alpha  - k_1) + \beta V(k_1) \bigg\} \\
	&= \max_{ k_1} \, \bigg\{ u(k_0^\alpha  - k_1) + \beta \bigg( \max_{ k_2} \, \bigg\{ u(k_1^\alpha  - k_2) + \beta V(k_2) \bigg\} \bigg) \bigg\} \\
	&= \max_{ k_1, k_2} \, \bigg\{ u(k_0^\alpha  - k_1) + \beta u(k_1^\alpha  - k_2) + \beta^2 V(k_2) \bigg\} \\
	&= \max_{ k_1, k_2, k_3} \, \bigg\{ u(k_0^\alpha  - k_1) + \beta u(k_1^\alpha  - k_2) + \beta^2 u(k_2^\alpha  - k_3) + \beta^3 V(k_3) \bigg\} \\
	&\;\; \vdots \\
	&= \max_{ \{ k_{t+1} \}_{t=0}^\infty } \, \bigg\{  \sum_{t = 0}^\infty \beta^t u(k_t^\alpha  - k_{t+1}) \bigg\} 
\end{align*}

\item Stokey and Lucas Thm 4.3: Sufficient condition is that $\lim_{n \to \infty} \beta^n V(k_n) = 0$ for all feasible sequences of $\{ k_t \}$ 

\end{witemize}
\end{frame}


%%%%%%%%%%%%%%%%%%%%%%%%%%  SLIDE   %%%%%%%%%%%%%%%%%%%%%%%%%%%%%%%%
\begin{frame}{2. First-Order and Envelope Conditions}
\begin{witemize}
\item How do we find optimal behavior? 

\item First-order condition for $k'$ is
\begin{equation*}
	\frac{\partial u(k^\alpha - k')}{\partial k'} = \beta \frac{\partial V(k')}{\partial k'}
\end{equation*}

\item FOC defines \textbf{policy function} $k'(k)$, use to plug back into Bellman:
\begin{align*}
	V(k) &= u(k^\alpha  - k'(k)) + \beta V(k'(k))
\end{align*}

\item The implied \textbf{consumption policy function} is: $c(k) = k^\alpha - k'(k)$

\item Important: we now characterize the solution to the model via \textbf{functions} ($V(k)$ and $c(k)$) and no longer via stochastic processes ($\{c_t\}_{t = 0}^\infty$)
\end{witemize}
\end{frame}


%%%%%%%%%%%%%%%%%%%%%%%%%%  SLIDE   %%%%%%%%%%%%%%%%%%%%%%%%%%%%%%%%
\begin{frame}{}
\begin{witemize}
\item Envelope theorem (in discrete time):
\begin{align*}
	\frac{\partial V(k)}{\partial k} &= \frac{\partial u(k^\alpha - k')}{\partial k}
\end{align*}

\item Work out at home: (i) prove envelope condition (ii) show resulting Euler equation coincides with solving sequence problem using Lagrangian
\end{witemize}
\end{frame}



%%%%%%%%%%%%%%%%%%%%%%%%%%  SLIDE   %%%%%%%%%%%%%%%%%%%%%%%%%%%%%%%%
\begin{frame}{3. Solving the Bellman equation with guess-and-verify}
\begin{witemize}
\item Assume the functional form $u(c_t) = \log(c_t)$ 

\item Guess that the value function takes the form 
\begin{equation*}
	V(k) = A + B \log(k)
\end{equation*}

\item Strategy: Plug into Bellman equation, match coefficients

\item Oftentimes, value function inherits functional form / shape of utility function
\end{witemize}
\end{frame}


%%%%%%%%%%%%%%%%%%%%%%%%%%  SLIDE   %%%%%%%%%%%%%%%%%%%%%%%%%%%%%%%%
\begin{frame}{}
\begin{witemize}
\item First-order condition for capital with log utility:
\begin{equation*}
	\frac{1}{k^\alpha - k'} = \beta B \frac{1}{k'} \implies  k' = \frac{\beta B}{1 + \beta B} k^\alpha
\end{equation*}

\item Plug guess into Bellman:
\begin{align*}
	V(k) &= \max_{k'} \, \bigg\{ \log(k^\alpha  - k') + \beta V(k') \bigg\} \\
	A + B \log(k) &= \max_{k'} \, \bigg\{ \log(k^\alpha  - k') + \beta A + \beta B \log(k') \bigg\} \\
	A + B \log(k) &= \log\bigg(k^\alpha  - \frac{\beta B}{1 + \beta B} k^\alpha \bigg) + \beta A + \beta B \log\bigg( \frac{\beta B}{1 + \beta B} k^\alpha \bigg)
\end{align*}

\item Solve for $A$ and $B$ so that coefficients of $\log(k)$ and constant terms cancel 

\end{witemize}
\end{frame}



%%%%%%%%%%%%%%%%%%%%%%%%%%  SLIDE   %%%%%%%%%%%%%%%%%%%%%%%%%%%%%%%%
\begin{frame}{}

More general Stokey-Lucas notation: Let $F(x_t, x_{t+1})$ be flow payoff

\vspace{8mm}
\textbf{Sequence problem}: Find $V(x)$ such that 
\begin{equation*}
	V(x_0) = \sup_{ \{ x_{t+1} \}_{t=0}^\infty} \, \sum_{t=0}^\infty \beta^t F(x_t, x_{t+1})
\end{equation*}
subject to $x_{t+1}$ in some feasible set $\Gamma(x_t)$, with $x_0$ given

\vspace{8mm}
\textbf{Bellman equation}: Find $V(x)$ such that 
\begin{equation*}
	V(x) = \sup_{x' \in \Gamma(x)} \Big\{ F(x, x') + \beta V(x') \Big\}
\end{equation*}
for all $x$ in the state space

\end{frame}



%%%%%%%%%%%%%%%%%%%%%%%%%%  SLIDE   %%%%%%%%%%%%%%%%%%%%%%%%%%%%%%%%
\begin{frame}{4. Bellman operator}
\begin{witemize}
\item In discrete time, Bellman equation is a \textbf{functional equation}

\item Define the \textbf{Bellman operator} $B$, operating on function $w(\cdot)$, as
\begin{equation*}
	(Bw)(x) \equiv \max_{x'} \Big\{ F(x, x') + \beta w(x') \Big\} \quad \text{ for all } x
\end{equation*}

\item \textbf{Operators} are maps from one function space to another (learn functional analysis!)

\item The value function $V(\cdot)$ is a fixed point of the operator $B$:
\begin{align*}
	(BV)(x) &= \max_{x'} \Big\{ F(x, x') + \beta V(x') \Big\} \\
	&= V(x)
\end{align*}

\end{witemize}
\end{frame}



%%%%%%%%%%%%%%%%%%%%%%%%%%  SLIDE   %%%%%%%%%%%%%%%%%%%%%%%%%%%%%%%%
\begin{frame}{}
\begin{witemize}

\item This idea is useful in numerical analysis: Suppose we guess $V^0$ and look for fixed point 
\begin{equation*}
	\lim_{n \to \infty} B^n V^0 = V
\end{equation*}

\item The Bellman operator is a \textbf{contraction mapping} under some conditions

\item This tells us that the Bellman operator converges (and we can use this to construct numerical fixed point algorithms)

\item For example: If an operator $B$ maps a complete metric space into itself and is a contraction mapping, then it has a unique fixed point

\item This is a very important idea. And you should read about this on your own. 

\end{witemize}
\end{frame}



%%%%%%%%%%%%%%%%%%%%%%%%%%  SLIDE   %%%%%%%%%%%%%%%%%%%%%%%%%%%%%%%%
\begin{transitionframe}
	{\color{white} \Huge \textbf{Part 3: Competitive Equilibrium} \vspace{2mm}}
\end{transitionframe}


%%%%%%%%%%%%%%%%%%%%%%%%%%  SLIDE   %%%%%%%%%%%%%%%%%%%%%%%%%%%%%%%%
\begin{frame}{1. Environment}

\begin{witemize}
\item Time is discrete with $t = 0, 1, \ldots$ and there is no uncertainty
\end{witemize}

\vspace{4mm}
\textbf{Households.} Consider a \textbf{representative household} with preferences
\begin{equation*}
	\max_{ \{ c_t \}_{t=0}^\infty } \sum_{t = 0}^\infty \beta^t u(c_t)
\end{equation*}

\begin{witemize}
\item $u(\cdot)$ is \textbf{flow utility}: units of $c_t$ ``dollars'' or ``apples'', units of $u(c_t)$ ``utils''

\item Household owns capital and faces budget constraint
\begin{equation*}
	k_{t+1} = i_t + (1 - \delta) k_t 
	\quad\quad \text{ and } \quad\quad
	c_t + i_t = r_t k_t + w_t
\end{equation*}
where $r_t$ is rental rate of capital, $w_t$ is wage

\item Household endowed with 1 unit of time, inelastically supplied as labor

\end{witemize}
\end{frame}



%%%%%%%%%%%%%%%%%%%%%%%%%%  SLIDE   %%%%%%%%%%%%%%%%%%%%%%%%%%%%%%%%
\begin{frame}{}

\vspace{4mm}
\textbf{Firms.} Consider a \textbf{representative firm} that operates technology 
\begin{equation*}
	y_t = f(k_t, \ell_t)
\end{equation*}
and earns profit $\Pi_t = y_t - r_t k_t - w_t \ell_t$ 

\vspace{5mm}
\begin{witemize}
\item \textbf{Assumptions:} $f(\cdot)$ is constant-returns and firms are perfectly competitive

\item Implies 0 profits and 
\begin{equation*}
	r_t = f_k(k_t, \ell_t) 
	\quad\quad \text{ and } \quad\quad
	w_t = f_\ell(k_t, \ell_t)
\end{equation*}

\end{witemize}
\end{frame}


%%%%%%%%%%%%%%%%%%%%%%%%%%  SLIDE   %%%%%%%%%%%%%%%%%%%%%%%%%%%%%%%%
\begin{frame}{}

\vspace{4mm}
\textbf{Market clearing.} How many markets are there? 
\begin{align*}
	y_t &= c_t + i_t \\
	\ell_t &= 1 
\end{align*}
What about capital?

\vspace{5mm}
\textbf{Definition.} Given initial capital $k_0$, a \textbf{competitive equilibrium} comprises an allocation $\{y_t, \ell_t, k_{t+1}, c_t, i_t\}$ and prices $\{w_t, r_t\}$ such that
\begin{enumerate}
	\item Households solve: taking as given $\{r_t, w_t\}$, $\max_{\{c_t\}} \sum_{t = 0}^\infty \beta^t u(c_t)$ subject to $k_{t+1} = i_t + (1 - \delta) k_t$ and $i_t + c_t = r_t k_t + w_t$ as well as $c_t, k_{t+1}, i_t \geq 0$

	\item Firms solve: taking as given $\{r_t, w_t\}$, $\max_{k_t, \ell_t} y_t - r_t k_t - w_t \ell_t$ subject to $y_t = f(k_t, \ell_t)$ and $k_t, \ell_t \geq 0$

	\item Markets for goods, labor and capital clear
\end{enumerate}

\end{frame}


%%%%%%%%%%%%%%%%%%%%%%%%%%  SLIDE   %%%%%%%%%%%%%%%%%%%%%%%%%%%%%%%%
\begin{frame}{2. Discussion}

\begin{witemize}
\item Make it a habit to always define equilibrium rigorously

	{\footnotesize Save yourself from countless mistakes and help others understand what you do}

\item One takeaway from this course: ``best-practice'' for setting up and presenting models

\item In macro, we write our models in terms of agents making decisions, then aggregate up 

	{\footnotesize Set up + present each agent's problem rigorously (Lucas Critique, ``microfoundations''}

\item Rule of thumb: get to your equilibrium definition as quickly as possible; 

	{\footnotesize Keep intermediate results and digressions for after}

\item Planning problem vs. competitive equilibrium?
\end{witemize}
\end{frame}
\end{document}
